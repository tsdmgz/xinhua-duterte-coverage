\chapter{Methodology}\label{chap:methodology}

A brand new Facebook account was made for the sole purpose of data
gathering. It is a necessity to start with a clean profile to eliminate 
interference from an already-established profile and to minimize
whatever influence Facebook's guess on a user's demographics is.

%something something keeps their algo and profile clean and keeps their
%envoronment sane. don't want to pollute my profile and to start with
%something as close and as clean as possible

Facebook provides a dedicated section for public profiles and interests
called ``Facebook Pages''. These pages 

In Facebook parlance, to follow a corporation or a personality is to
``Like'' their page. This action tells Facebook that this user has an
interest in whatever the page is invested on such as a specific media
outlet, a hobby, or whatever interest the page presents itself to be and
wishes to be notied via their feed on any updates a page generates.

\section{Limitations}\label{sec:limits}

When working with social media, one must remember that every click,
movement, and words typed while within the network will influence what
kind of content will be presented to the user.

Due to the volume of data generated by users of the site, the filter is
necessary. Operating without the filter would make sifting through data
tedious, extremely resource heavy, and time consuming--even with
automated means. When operating with the filter, we are returned
meaningful data at the cost of visibility. It is then impossible to see
all the data generated for a type of content because we are only given
data to what Facebook's algorithm sees as content relevant to us.

In any case, we are bound to Facebook's terms making it is impossible to
turn off the filter for us. Therefore it must be recognized that content
presented here may or may not be possible to reproduce due to several
factors such as, but not limited to time, account status, a user's
interests as seen by Facebook, location, and age.

% vim: tw=72 smartindent breakindent syntax=tex
