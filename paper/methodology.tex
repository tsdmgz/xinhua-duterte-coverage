\section{Methodology}\label{sec:methodology}

The sample dataset was built by making a search at Xinhua's English
language website at \url{http://www.xinhuanet.com/english/} for the
string ``Duterte'' and downloading all the news articles returned by the
result pages. This search string forms the basis of the whole dataset.
The latest news article is on November 11, 2016.

Articles that had the word ``Duterte'' but did not specifically refer to
Rodrigo Duterte were removed from the analysis dataset. Articles that
were included in the results but did not mention Duterte on the first
page are also pruned. There are also ``news summaries'' present in the
download, summarizing the week's world news in a single article. Those
items were also pruned. Despite the removal of these articles, I have
not removed them from the download for completeness.

To make this possible, a routine was performed to automate the
download of news articles returned by the search operation. A small
script (see appendix \ref{adx:python-script}) was written to automate
the cleanup and extraction of the news article full text. Part of the
data extracted includes filename, date of publication, title, and
body text.

In analyzing the data, a simple thematic analysis was performed. News
articles were read individually and then tagged according to a general
theme.
%FIXME

In a second analysis, RapidMiner 7.3 was used for text mining in all
articles included in the dataset. A word count was performed and
n-grams were generated.

All assets used in data gathering, results, and a description of the
working environment are available in a GitHub repository at
\href{https://github.com/tsdmgz/xinhua-duterte-coverage}{github.com/tsdmgz/xinhua-duterte-coverage}.

\section{Limitations}\label{sec:limits}

For manageability, I have limited the data to the following events: the
Hague Ruling on the South China Sea dispute, and President Duterte's
first state visit to China. These were chosen because these two events
carry significant weight in China-Philippines bilateral relations.
However, the articles present in the dataset is not comprehensive as
these articles were only present because President Duterte's name was
mentioned at least once.

% vim: tw=72 smartindent breakindent syntax=tex
