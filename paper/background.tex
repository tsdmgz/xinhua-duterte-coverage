\section{Background}\label{sec:background}

News has considerable influence on public opinion. So much so that
countries actively censor at most, and downplay at the very least, news
that is not in their favor. And a country very well-known to employ
this practice is China, among other others. And China is no stranger to
this game.

Given the notoriety of the Chinese media, citizens are skeptical by
default by the news given out by state outlets and heavily rely on
social media for news\autocite{xu_watching_2016}. In addtion, some
articles may not be permitted to be published or even be retroactively
pulled due to the following reasons:
\begin{enumerate*}[label={\alph*}.)]
	\item it is contrary to the basic principles that are laid down
	in the Constitution, laws, or administration regulations;
	\item is seditious to the ruling regime of the state or the
	system of socialism;
	\item subverts state power or sabotages the unity of the state;
	\item incites ethnic hostility or racial discrimination, or
	disrupts racial unity;
	\item spreads rumors or disrupts social order
	\item propagates feudal superstitions; disseminates obscenity,
	pornography, or gambling; incites violence, murder, or terror;
	instigates others to commit offences;
	\item publicly insults or defames others;
	\item harms the reputation or interests of the state;
	\item has content prohibited by laws or administrative
	regulations.
\end{enumerate*} \autocite[107]{liang_internet_2010}. These broad
guideines are made possible by the equally broad Article 28 of the
Chinese consititution which states ``The state maintains public order
and suppresses treasonable and other counter revolutionary activities;
it penalizes actions that endanger public security and disrupt the
socialist economy and other criminal activites, punishes and reforms
criminals.''

Given that, this study attempts to look into how the English language
edition of Xinhua covered then president-elect Rodrigo Duterte one week
before and after his inauguration, and one week before and after the
landmark Hague Ruling on the South China Sea issue.

The reasons for selecting Xinhua English as the subject in this study
is twofold. One, Xinhua English was chosen due to language limitations.
While it is highly desirable to be able to cover the Chinese edition,
my command of Mandarin is not enough to be able to read and comprehend
the articles. Two, given that language limits and defines the audience
of the work, this would imply that the articles presented there are for
the international audience at the most, and to the English-speaking
community at the least. In a way, these articles may telegraph China's
stance on certain topics, at least how they would want to present it to
the rest of the English-speaking audience.

% vim: tw=72 smartindent breakindent syntax=tex
