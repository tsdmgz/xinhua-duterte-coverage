\section{Data and Analysis}\label{sec:data}

From the retrieved articles, I have chosen to look into two events,
including the week before and after to complete the coverage: the Hague
Tribunal ruling on the South China Sea dispute, and President Duterte's
first state visit to China. These two events were chosen because of the
significance of these events to both China and The Philippines and
received extensive coverage among major news outlets. Further, these
events usually go back to the president for statements or for guidance
on what the next course of action will be.

\subsection{Hague Ruling on the South China Sea}

China has long made their discontent highly visible on the South China
Sea
disputes\autocites{sun_new_philippine_president_ties_2016}{li_end_scs_farce_2016}{xinhua_us_cold_war_not_solution_2016}.
Their discontent has reached a record low when the arbitration ruling
came out on July 12, 2016. A series of articles and editorials calling
out the ruling as ``illegal'' and farcial to say the least, calling it
destabilizing, ``simply bad'', and ``invalid'' at most, were published
on the same day. This pattern of publishing continued for six
consecutive days.

This pattern of publishing editorials and news articles about a
sensitive issue, specially when it upsets a nation is not new and
unusual in itself. What is noteworthy here, however, is the strong will
to be able to publish several editorials, and even more so when it was
done over consecutive days.

A second pattern among the articles emerged. Aside from editorials,
there were interviews with field experts and political personalities
from other countries. These experts and personalities are, but not
limited to Brazil, Argentina, Mexico, India, South Africa, and Czech
Republic. The common theme among these countries is that they are
strong trade partners of China, either as BRICS, among others.

Going further, the interviews make common points:
\begin{enumerate*}[label={\alph*}.)]
	\item the ruling is illegal and/or baseless,
	\item there is a call to to forgo the ruling and go
	back to the negotiation table specifically, as a bilateral
	affair,
	\item the ruling solves nothing, and
	\item the ruling is out of bounds and/or excessive
\end{enumerate*}

\subsection{First state visit to China}

The general theme throughout the whole period, unsurprisingly revolves
around foreign policy and occasionally touches on the South China Sea
row. Interestingly, there is a neutral to positive portrayal of the
issue when referring to Duterte. This does not mean the issue can be
resolved soon, but there is indication of cautious optimism between
China and the Duterte administration. Conversely, whenever the issue is
raised when associated with either the United States and/or former
president Benigno Aquino, III, there is strong indication of aversion
and recoil.

As noted in the news release pattern during the Hague ruling, a series
of commentaries and field interviews were published on the conclusion
of the state visit. One was from the Lao Republic praising the
procedings as a positive step forward to putting relations back on
track. In addition, Xinhua's coverage on President Duterte followed him
to Japan and as expected, the topic of comfort women was brought up.
An editorial published advising Japan to promote peace within the
Asia-Pacific region and to follow the example the Philippines made by
reaching a ``peaceful consensus'' and to ``stop muddying the waters in
the South China Sea''\autocite{chen_commentary_2016}.

These portrayals are not a surprise, given that China and among Chinese
commentators in repeated statements, have referred to the arbitration
action as an ``abuse of international law'' by the Aquino administration
and a
``farce''\autocites{sun_new_philippine_president_ties_2016}{li_end_scs_farce_2016}{xinhua_us_cold_war_not_solution_2016}.
Further, the United States is frequently named as a meddler within the
region's affairs and treats their presence as a threat to China.

Beyond the sea row, their response to President Duterte ranges from
again, cautiously optimisic to positive. China is expressing strong
support for his War on Drugs and is willing to lend expertise,
equipment, training, and cross-border enforcement to make it happen.

On the restoration of friendly China-Philippine relations, there are
several references to the millenia-long shared contact between China and
the Philippines. Present with those words is a feeling of longing to go
back to that previous state of relations.

\subsection{General remarks}

Whenever there is strong sentiment about a topic, Xinhua will publish a
string of articles and editorials up to a span of several days as seen
from their reaction to the arbitration decision. Among these
commentaries, several of these writings are unattributed to an
individual, but is generally attributed to Xinhua itself.

Second, there is specific and consistent use of vocabulary. Calling the
arbitration procedings a ``farce'', consistent calls for bilateral
meetings, frequently associating the Aquino administration as
irresponsible and reckless, farcial, and the root cause of
deteriorating Philippine-China relations.

From the mined texts, a common phrase that frequently appears between
the two events relates to the South China Sea. As from the previous
assessment, the phrase carries different weights between the two
events. Former president Noynoy Aquino was frequently referenced up to
14 times during the height of the Hague ruling, associated with
negative impressions, and current president Rodrigo Duterte as a
positive impression.

% vim: tw=72 smartindent breakindent syntax=tex
